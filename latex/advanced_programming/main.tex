\documentclass[11pt,a4paper]{article}

% ----- Useful packages -----
\usepackage[utf8]{inputenc}           % allows UTF-8 characters
\usepackage[T1]{fontenc}              % better font output
\usepackage[english,italian]{babel}   % language
\usepackage{amsmath, amssymb}         % math
\usepackage{amsthm}                   % theorem/definition
\usepackage{hyperref}                 % links inside the PDF
\usepackage{enumitem}                 % let you change list arrow
\usepackage{geometry}                 % margins
\usepackage{subcaption}               % align images
\usepackage{graphicx}                 % images loader
\usepackage{fontspec}                 % change font
\geometry{a4paper, margin=2.5cm}
%\setmainfont{Noto Sans}

% --- Style section ---
\usepackage{titlesec}
\titleformat{\section}{\large\bfseries}{\thesection}{1em}{}
\titleformat{\subsection}{\normalsize\bfseries}{\thesubsection}{1em}{}

% --- Env settings ---
\newtheorem{theorem}{Theorem}[section]
\newtheorem{lemma}[theorem]{Lemma}
\newtheorem{proposition}[theorem]{Proposition}
\newtheorem{corollary}[theorem]{Corollary}

\theoremstyle{definition}
\newtheorem{definition}[theorem]{Definition}

\theoremstyle{remark}
\newtheorem{observation}[theorem]{Observation}

% --- Info documento ---
\title{
  {
    \Large
    \bfseries{University of Pisa}
  }

  Advanced Programming
}
\author{Enrico Fiasco}
\date{Year 2025}

\begin{document}

\maketitle
\tableofcontents
\newpage

\section{Capitolo 1}
Breve introduzione al contenuto del capitolo.

\subsection{Concetto principale}
Qui scrivi il riassunto.

\begin{definition}
  Una definizione importante spiegata in breve.
  Un teorema fondamentale con eventuale formula:
  \[
    a^2 + b^2 = c^2
  \]
\end{definition}

\section{new}
Appunti personali, collegamenti o osservazioni pratiche.

\subsection{Lista di punti chiave}
%\begin{itemize}[label=.]
\begin{itemize}
  \item Punto 1
  \item Punto 2
  \item Punto 3
\end{itemize}

\section{Capitolo 2}
Altri riassunti...

\end{document}
